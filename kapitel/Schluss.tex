
\section{Schluss}
\label{sec:conclusion}
In dieser Arbeit haben wir zwei erweiterte Ansätze des \emph{Reinforcemnt Learning} vergleichend gegenübergestellt. Wir konnten zeigen, dass beide Konzepte, trotz ihrer theoretischen Differenzen, in gewissen Bereichen eine durchaus nicht unähnliche Funktionsweise aufweisen. Insbesondere findet sowohl das \emph{Curiosity}-basierte Modell, als auch der \emph{Diversity}-Ansatz eine adäquate Lösung für das in Kapitel \ref{sec:sparse_reward} vorgestellte \emph{Sparse Reward Problem}, indem beide nicht auf externe Belohnungen angewiesen sind.


Wir sehen in beiden Ansätzen großes Potential. Vor allem das noch nicht häufig thematisierte Vorgehen des Erlernens von Fähigkeiten mittels Diversity sollte weiter erforscht werden.