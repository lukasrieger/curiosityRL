% +------------------------------------------------------------------------------+ %
% | EDIT THESE SETTINGS ACCORDING TO YOUR THESIS                                 | %
% +------------------------------------------------------------------------------+ %

\newcommand{\seminar}{Trends in Mobilen und Verteilten Systemen} % comment for master seminar
%\newcommand{\seminar}{Vertiefte Themen in Mobilen und Verteilten Systemen} % uncomment for master seminar

\newcommand{\authorA}{David Jonathan Müller}
\newcommand{\authorB}{Lukas Johannes Rieger}
\newcommand{\supervisor}{Thomas Gabor}

\newcommand{\thesistitle}{Curiosity und Diversity im Kontext von Reinforcement Learning -- Ein Vergleich}

\newcommand{\thesisabstract}{% \/ put your thesis abstract below \/
Ansätze, die das traditionelle Reinforcement-Learning um zusätzliche \emph{interne} Belohnungen erweitern, stellen in vielen Anwendungsfällen die deutlich effektivere Wahl dar. In dieser Arbeit stellen wir insbesondere den Ansatz des \emph{Curiosity}-basierten Reinforcement-Learning vor. Wir erklären dessen theoretische Grundlagen, wie sie nach \citeauthor{curiosity_schmidhuber} erläutert werden. Anschließend stellen wir einen \emph{Diversity}-basierten Reinforcement-Learning Ansatz vor und stellen ihn \citeauthor{curiosity_schmidhuber}s Ansatz vergleichend gegenüber. Wir zeigen, wie beide Konzepte es einem RL-Agenten erlauben, auch in Umgebungen mit spärlichen externen Belohnungen zu agieren. Weiter legen wir dar, wie beide Ansätze trotz unterschiedlicher Herangehensweisen gewisse operationale Gemeinsamkeiten in Hinsicht auf die Selektion zukünftiger Aktionen aufweisen. 

}% <-- mind this closing brace!

\newcommand{\thesisauthorship}{% \/ describe who wrote what below \/
David Jonathan Müller hat die Abschnitte~\ref{sec:basics} und \ref{sec:diversity} verfasst. Lukas Johannes Rieger hat den Abschnitt~\ref{sec:Curiosity_Schmidhuber} verfasst. Die Abschnitte~\ref{sec:intro}, \ref{sec:comparison}, \ref{sec:related} und \ref{sec:conclusion} haben beide Autoren gemeinsam verfasst.
}% <-- mind this closing brace!s

\newcommand{\smallspace}{\vspace{4mm}}
\newcommand{\bigspace}{\vspace{7mm}}
\newcommand{\source}[1]{\caption*{Quelle: {#1}}}


\selectlanguage{ngerman} %options: english, ngerman
 